%laden der Präambel mit Latexbefehlen/-klassen
% !TeX encoding = UTF-8
% !TeX spellcheck = de_DE

%Dokumentklasse, Spracheinstellung, Schriften
\documentclass[a0paper,portrait]{baposter}
\usepackage[ngerman]{babel}
\usepackage[utf8]{inputenc}
\usepackage{arev}
\usepackage[T1]{fontenc}

%Mathe, Symbole, Einheitendarstellung, Chemie
\usepackage{amsmath}
\usepackage{amsxtra}
\usepackage{eurosym}
\usepackage{siunitx}  
\sisetup{locale=DE}
\usepackage[version=4]{mhchem}

%Typographie
\usepackage[auto]{microtype}

%Einbindung von Bildern, Tabellen, pdf-Seiten, Quellcode
\usepackage{booktabs}
\usepackage{multirow}
\usepackage{paralist}

%Darstellung der Literaturangaben
\usepackage[
backend=biber,
style=science,
citestyle=numeric-comp,
sorting=none,
maxbibnames=1,
firstinits=true
]{biblatex}

\bibliography{literatur/refs}
\setlength\bibitemsep{2.5pt}

%Definition der Farben
\definecolor{standardfontcolor}{RGB}{0,0,0} 
\definecolor{bordercol}{RGB}{113,113,113}
\definecolor{headercol1}{RGB}{255,255,255}
\definecolor{headercol2}{RGB}{113,113,113}
\definecolor{headerfontcol}{RGB}{0,0,0}
\definecolor{boxcolor}{RGB}{255,255,255}

\definecolor{darkgreen}{cmyk}{0.2394,0.0000,0.2394,0.2627}
\definecolor{orange}{cmyk}{0.0000,0.7294,1.0000,0.0000} 
\definecolor{floralwhite}{cmyk}{0.0000,0.0196,0.0588,0.0000}
\definecolor{gray}{cmyk}{0.0000,0.0000,0.0000,0.0392}
\definecolor{lightgreen}{cmyk}{0.7554,0.0000,0.7554,0.4549}
\definecolor{lightblue}{rgb}{0.145,0.6666,1} % Defines the color used for content box headers

\begin{document}
	\typeout{Poster rendering started}
	
\background{
	\begin{tikzpicture}[remember picture,overlay]%
	\draw (current page.north west)+(-2em,2em) node[anchor=north west]
	{\includegraphics[height=1.1\textheight]{figures/background}};
	\end{tikzpicture}
}

\color{standardfontcolor}

\begin{poster}{
		grid=false,
		columns=2,
		%colspacing=length
		headerheight=0.125\textheight,
		eyecatcher=false, 
		borderColor=darkgreen,
		headerColorOne=darkgreen,
		headerColorTwo=lightgreen,
		headerFontColor=black,
		% Only simple background color used, no shading, so boxColorTwo isn't necessary
		boxColorOne=boxcolor,
		headershape=roundedright,
		headerfont=\sffamily\bfseries\Large,
		textborder=rectangle,
		headerborder=open,
		boxshade=plain,
		background=none
%		background=user
	}
	%%% Eye Cacther %%%%%%%%%%%%%%%%%%%%%%%%%%%%%%%%%%%%%%%%%%%%%%%%%%%%%%%%%%%%%%%
	{
		Eye Catcher, empty if option eyecatcher=false - unused
	}
%----------------------------------------------------------------------------------------
%	TITLE AND AUTHOR NAME
%----------------------------------------------------------------------------------------
%
{
	\textsf %Sans Serif
	{\Large \textsc{Guarnary}: Mitigating Buffer Overflow Using \break Hardware Assisted Virtualization Features}
}
{\sf\vspace{0.5em}\\
	Pinky*, Brain und Jane Doe
	\vspace{0.1em}\\
	\small{Technische Universität Berlin, Fakultät III, Vertiefendes Rechnerpraktikum zur Energietechnik
	\vspace{0.2em}\\
	* pinky@campus.tu-berlin.de}
}
% University/lab logo
{\includegraphics[height=3cm]{figures/labri.jpg}} 
%
%
\headerbox{Aufgabenstellung}{name=einleitung,column=0,row=0,span=2}
{
Vorlage für die Erstellung eines Posters. Es wird die Klasse baposter verwendet. In der entsprechenden Dokumentation finden sich Hinweise zur Gestaltung. Die Vorlage bietet ein einfaches Layout, dass zur Bearbeitung der Aufgabenstellung ausreichend sein sollte. Die Einbindung verschiedener Elemente ist beispielhaft gezeigt.

Hier könnte die Einleitung stehen, also Motivation, Ziel der Arbeit, Aufgabenstellung, etc.

Literaturverweise, ein Buch \cite{epple2009,gruhn1976} und ein Artikel \cite{Klaucke2020,Hofmann2018}
}

\headerbox{Fließbild}{name=fliessbild,column=0,row=0,below=einleitung,span=2}
{	
    \begin{center}
    \label{tab:vorgaben}
    \begin{tabular}{lllc}
        \toprule
        Parameter & Symbol & Einheit & Wert \\
        \midrule
        Umgebungstemperatur & $T_0$ & \si{\celsius} & 25 \\
        Umgebungsdruck & $p_0$ & \si{\bar} & 1 \\
        Nettoleistung & $\dot W_\text{netto}$ & \si{MW} & 30 \\
        Massenstrom, Dampf & $\dot m_9$ & \si{kg/s} & 14 \\
        Druck des Sattdampfes & $p_{11}$ & \si{bar} & 20 \\
        \bottomrule    
    \end{tabular}
    \end{center}
}

\headerbox{Vorgaben}{name=vorgaben,column=0,below=fliessbild,span=2}{
Die Tabelle fasst relevante vorgegebene Größen zusammen.

	\begin{center}
	\label{tab:vorgaben}
	\begin{tabular}{lllc}
			\toprule
			Parameter & Symbol & Einheit & Wert \\
			\midrule
			Umgebungstemperatur & $T_0$ & \si{\celsius} & 25 \\
			Umgebungsdruck & $p_0$ & \si{\bar} & 1 \\
			Nettoleistung & $\dot W_\text{netto}$ & \si{MW} & 30 \\
			Massenstrom, Dampf & $\dot m_9$ & \si{kg/s} & 14 \\
			Druck des Sattdampfes & $p_{11}$ & \si{bar} & 20 \\
			\bottomrule    
	\end{tabular}
    \end{center}

Wichtige Annahmen und Vereinfachungen:

\begin{compactitem}
    \item Stationärer Prozess
    \item Alle Komponenten nach außen adiabat
    \item Druckverluste vernachlässigt
    \item Gute Laune!
    \item Gruppenarbeit macht Spaß.
\end{compactitem}
}

\headerbox{Ebsilon}{name=ebsilon,column=0,below=vorgaben,span=2}{
Ebsilon toll. Oberfläche toll. Implementierung toll.

Punkte, die beim Modellieren und Simulieren nicht so toll sind:

\begin{compactitem}
    \item kein mp3-plugin
    \item keine automatische Modellerstellung
    \item \dots
    \item jetzt fällt mir nix mehr ein. 
\end{compactitem}
}


\headerbox{Ergebnis, Fazit}{name=ergebnis,column=0,below=ebsilon,span=1}{
Das haben wir herausgefunden. Das ist wichtig:
\begin{compactitem}
	\item erster Punkt
	\item zweiter Punkt
	\item dritter Punkt
	\item ...
\end{compactitem}
}

\headerbox{Literatur}{name=literatur,column=1,below=ebsilon,span=1}
{
\AtNextBibliography{\footnotesize}
\setlength\bibitemsep{0pt}
\printbibliography[heading=none]
}

\end{poster}
\end{document}