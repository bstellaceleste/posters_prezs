%----------------------------------------------------------------------------------------
%	PACKAGES AND THEMES
%----------------------------------------------------------------------------------------
\documentclass[aspectratio=169,xcolor=dvipsnames]{beamer}
\usetheme{SimplePlus}

\usepackage{hyperref}
\usepackage{graphicx} % Allows including images
\usepackage{booktabs} % Allows the use of \toprule, \midrule and \bottomrule in tables
\usepackage[utf8]{inputenc}
%\usepackage[T1]{fontenc}
\usepackage[french]{babel}
\usepackage{eurosym}
% \addtobeamertemplate{navigation symbols}{}{%
%     \usebeamerfont{footline}%
%     \usebeamercolor[fg]{footline}%
%     \hspace{1em}%
%     \insertframenumber/\inserttotalframenumber
% }
%----------------------------------------------------------------------------------------
%	TITLE PAGE
%----------------------------------------------------------------------------------------

\title[YCOWS]{Sensibilisation des Acteurs de l'Enseignement Secondaire et Supérieur sur le Harcèlement des Jeunes Filles en Milieu Académique\\} % The short title appears at the bottom of every slide, the full title is only on the title page

\subtitle{YCOWS: Young Committed Women Scientists}

\author[Stella Bitchebe \& Alain Tchana] {Stella Bitchebe \& Alain Tchana}

\date{\today} % Date, can be changed to a custom date


%----------------------------------------------------------------------------------------
%	PRESENTATION SLIDES
%----------------------------------------------------------------------------------------

\begin{document}

\makeatletter
\setbeamertemplate{footline}
{
  \leavevmode%
  \hbox{%
  \begin{beamercolorbox}[wd=.333333\paperwidth,ht=2.25ex,dp=1ex,center]{title in head/foot}%
    \usebeamerfont{title in head/foot}\insertshorttitle\expandafter\beamer@ifempty\expandafter{\beamer@shortinstitute}{}{~~(\insertshortinstitute)}
  \end{beamercolorbox}%
  \begin{beamercolorbox}[wd=.333333\paperwidth,ht=2.25ex,dp=1ex,center]{date in head/foot}%
    \usebeamerfont{date in head/foot}\insertshortdate
  \end{beamercolorbox}%
  \begin{beamercolorbox}[wd=.333333\paperwidth,ht=2.25ex,dp=1ex,right]{date in head/foot}%
    \usebeamerfont{date in head/foot}\insertframenumber{}/\inserttotalframenumber\hspace*{2ex}    
    \hspace*{6ex}
  \end{beamercolorbox}}%
  \vskip0pt%
}
\makeatother

\begin{frame}[noframenumbering,plain]
    % Print the title page as the first slide
    \titlepage
\end{frame}


\begin{frame}{Origines et Porteurs du Projet}
    \begin{block}{Formations FWIS 2021}
        \begin{itemize}
            \item Reconnaître des situations de sexisme et de harcèlement
            \item Réagir face à de telles situations
            \item Textes de lois qui protègent les femmes face au harcèlement
            \item Devoirs de leurs supérieurs hiérarchiques
            \item Etc.
        \end{itemize}
    \end{block}
    \pause
    \begin{block}{YCOWS}
        Association \textit{Young Committed Women Scientists}\\
    \end{block}
\end{frame}

\begin{frame}[allowframebreaks]{Objectifs}
    \begin{enumerate}
        \item Attirer l’attention de tous les acteurs du milieu académique au Cameroun (secondaire, supérieur, structures de stage académique) sur les difficultés que rencontrent la jeune fille tout au long de sa scolarité
        \item Former élèves et étudiants, parents, enseignants et responsables d’établissements et de stages académiques sur les questions de harcèlement : identifier les causes, les conséquences et les différents types de violence et de harcèlement (violences physiques, psychologiques, verbales, etc.)
        \item Informer les élèves et étudiants sur leurs droits et les parents et enseignants sur leurs devoirs face aux plaintes
        \item Apprendre aux élèves et étudiants à éviter et réagir face à de telles situations : textes juridiques, lois et réglementations pertinentes pour répondre aux agressions, se prémunir des grossesses précoces et non désirées, savoir faire face à la pression des pairs
        \item Apprendre aux parents, enseignants et responsables académiques les moyens d’action dont ils disposent pour protéger les enfants : signaler les cas de violences en s’appuyant sur les personnes ressources et les services adéquats
        \item Proposer au corps administratif et enseignant des directives quant à la mise en place dans leurs différents établissements, de points focaux chargés de recensés et remonter les différentes plaintes des élèves et étudiantes
    \end{enumerate}
\end{frame}

%------------------------------------------------

\begin{frame}{Plan}    
    \begin{itemize}
        \item Conférences/Formations au moins annuelles
        \pause
        \item Vitrines numériques de l'association : Site web, pages LinkedIn, Youtube, etc. pour des vidéos de sensibilisation, partage d'expériences, etc.
        \pause
        \item Conception et mise sur pied d'une plateforme/forum : où les jeunes filles pourront s’exprimer et dénoncer librement et de façon anonyme sans craindre de représailles
        \item Contrats avec des avocats indépendants qui pourront défendre les jeunes filles gratuitement (parfois les frais juridiques découragent à entamer toute procédure de plainte ou de dénonciation)
    \end{itemize}
\end{frame}

%------------------------------------------------
\begin{frame}{Supports \& Besoins}
    \begin{enumerate}
        \item Aspect Juridique : Avocats et Juristes qui connaissent la réglementation camerounaise en matière de protection de la jeune fille et de la femme
        \item Aspect Logistique et Financier : Organisation des conférences, montage de la plateforme, rémunération des avocats et juristes, etc.
    \end{enumerate}
\end{frame}
%------------------------------------------------
\begin{frame}{Conférence Pilote}
    \begin{enumerate}
        \item Ville : Yaoundé pour la 1re édition. Le but étant de pouvoir étendre les formations de façon autonome à toutes les villes
        \item Etablissements : 3 à 4 établissements d'enseignement secondaire de la ville de Yaoundé et 3 à 4 établissements d'enseignement supérieur
        \item Formateurs : juristes sélectionnés, jeunes doctorantes, des enseignants et des parents d'élèves
        \item Date : ??
        \item Budget estimatif : XXXX \euro{} (en cours d'élaboration)
    \end{enumerate}
\end{frame}
%------------------------------------------------
\begin{frame}{Qu'attendons nous de la fondation ? : Association avec YCOWS pour porter le projet}
    \begin{enumerate}
        \item Soutien et accompagnement dans l'organisation : aucune expérience dans ce genre de projet
        \item Communication et diffusion
        \item Mise en relation avec des représentations d'ONG et d'organismes au Cameroun (si contacts il y en a)
        \item Appuis technique et financier
    \end{enumerate}
\end{frame}
%------------------------------------------------
\begin{frame}
    % Print the title page as the first slide
    \titlepage
\end{frame}
%----------------------------------------------------------------------------------------
-> définitions
-> textes de lois, les règles
-> associations/collectifs déjà présents (au lieu de juristes/avocats directement), acteurs
-> charte à diffuser
-> créer des supports que les établissements peuvent diffuser
-> clashes.fr
-> contacter des universités, chercher des ambassadeurs qui portent le msg
-> en plus de la plateforme mm "café forum"
-> commencer mm des webinaires en ligne sur fb par exple
\end{document}